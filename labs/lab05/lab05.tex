\documentclass[]{article}
\usepackage{lmodern}
\usepackage{amssymb,amsmath}
\usepackage{ifxetex,ifluatex}
\usepackage{fixltx2e} % provides \textsubscript
\ifnum 0\ifxetex 1\fi\ifluatex 1\fi=0 % if pdftex
  \usepackage[T1]{fontenc}
  \usepackage[utf8]{inputenc}
\else % if luatex or xelatex
  \ifxetex
    \usepackage{mathspec}
  \else
    \usepackage{fontspec}
  \fi
  \defaultfontfeatures{Ligatures=TeX,Scale=MatchLowercase}
\fi
% use upquote if available, for straight quotes in verbatim environments
\IfFileExists{upquote.sty}{\usepackage{upquote}}{}
% use microtype if available
\IfFileExists{microtype.sty}{%
\usepackage{microtype}
\UseMicrotypeSet[protrusion]{basicmath} % disable protrusion for tt fonts
}{}
\usepackage[margin=1in]{geometry}
\usepackage{hyperref}
\PassOptionsToPackage{usenames,dvipsnames}{color} % color is loaded by hyperref
\hypersetup{unicode=true,
            pdftitle={Lab 05},
            colorlinks=true,
            linkcolor=Maroon,
            citecolor=Blue,
            urlcolor=Blue,
            breaklinks=true}
\urlstyle{same}  % don't use monospace font for urls
\IfFileExists{parskip.sty}{%
\usepackage{parskip}
}{% else
\setlength{\parindent}{0pt}
\setlength{\parskip}{6pt plus 2pt minus 1pt}
}
\setlength{\emergencystretch}{3em}  % prevent overfull lines
\providecommand{\tightlist}{%
  \setlength{\itemsep}{0pt}\setlength{\parskip}{0pt}}
\setcounter{secnumdepth}{0}
% Redefines (sub)paragraphs to behave more like sections
\ifx\paragraph\undefined\else
\let\oldparagraph\paragraph
\renewcommand{\paragraph}[1]{\oldparagraph{#1}\mbox{}}
\fi
\ifx\subparagraph\undefined\else
\let\oldsubparagraph\subparagraph
\renewcommand{\subparagraph}[1]{\oldsubparagraph{#1}\mbox{}}
\fi

%%% Use protect on footnotes to avoid problems with footnotes in titles
\let\rmarkdownfootnote\footnote%
\def\footnote{\protect\rmarkdownfootnote}


  \title{Lab 05}
    \author{}
    \date{}
  

% change section title styling
\usepackage{sectsty}
\sectionfont{\normalsize\normalfont\itshape}
\subsectionfont{\normalsize\normalfont}

% use fancyhdr style
\usepackage{fancyhdr}
\pagestyle{fancy}
\fancyhead[LO, LE]{Lab 05}
\fancyhead[RO, RE]{GEOG 432/832}
\makeatletter
\renewcommand{\maketitle}{\bgroup\vspace*{-1cm}\setlength{\parindent}{0pt}
\begin{flushleft}
  \@author
  
  \@date
  
\end{flushleft}\egroup
}
\makeatother

\begin{document}
\maketitle

\hypertarget{lab-05-introduction-to-or-refresher-of-data-science-tools-in-python}{%
\section{Lab 05: Introduction to (or refresher of) data science tools in
Python}\label{lab-05-introduction-to-or-refresher-of-data-science-tools-in-python}}

\hypertarget{read-the-instructions-completely-before-starting-the-lab}{%
\subsubsection{Read the instructions COMPLETELY before starting the
lab}\label{read-the-instructions-completely-before-starting-the-lab}}

This lab is the first exercise in the second half of the class, which
focuses on spatial data science using open source Python tools. This lab
is designed to help you gain some basic familiarity with data structures
and functions, in particular the DataFrame.

This following tasks use the CSV file: \emph{lab05\_climate\_data.csv},
which can be found in the /labs/lab05/ directory of the Course GitHub
repository
(\url{https://github.com/pjbitterman/UNL_geog432/tree/main/labs/lab05}).
This file includes a limited set of \emph{daily} climate variables for
Lincoln, NE in 2020.

\begin{itemize}
\tightlist
\item
  month: numeric representation month
\item
  day: numeric day of month
\item
  year: 2020
\item
  day\_of\_week: String for the day of week of the observation
\item
  t\_max: maximum daily temperature
\item
  t\_min: minimum daily temperature
\item
  t\_mean: arithmatic mean daily temperature
\item
  precip: daily precipitation (a ``T'' value means ``trace
  precipitation'')
\end{itemize}

Using this file, complete the following tasks below. Show your work. All
results must be ``printed'' to the Jupyter notebook or to the console.
The in-class exercises and examples here:
\url{https://darribas.org/gds_course/content/bB/lab_B.html} will aid you
in your task. You may need to consult the help documentation for pandas
or matplotlib (or whichever packages you use) to achieve your tasks.
Note, you will need to \emph{programmatically} deal with trace
precipitation values.

\hypertarget{tasks}{%
\subsubsection{Tasks:}\label{tasks}}

\begin{enumerate}
\def\labelenumi{\arabic{enumi}.}
\item
  On what day of the year (yyyy-mm-dd) did the coldest minimum
  temperature (t\_min) occur? The second coldest?
\item
  How many days had measurable precipitation in July 2020?
\item
  What is the month with the highest \emph{average} maximum temperature
  (t\_max)?
\item
  What day of the week had the greatest \emph{total} precipitation in
  2020? The lowest \emph{total} precipitation?
\item
  Print the 7 days with the highest minimum temperature (t\_min)
\item
  Make a histogram of maximum daily temperature (t\_max)
\item
  Make a histogram of daily precipitation. Bonus points if you add a
  smoothing function (e.g., kernel density estimator) to the plot
\item
  Make a scatter plot with minimum daily temperature (t\_min) on the
  x-axis and daily precipitation on the y-axis. Make the points
  different colors by month. For example, January would plot as red,
  February as purple, etc.
\end{enumerate}

\hypertarget{questions}{%
\subsubsection{Questions:}\label{questions}}

\begin{enumerate}
\def\labelenumi{\arabic{enumi}.}
\item
  What previous experience (in Python or other languages) do/did you
  have with ``data science'' (broadly conceived)?
\item
  What new skills did you learn (or refresh) in this lab?
\item
  What did you find particularly challenging? How did you overcome
  this/these challenge(s)?
\end{enumerate}

\hypertarget{what-to-turn-in}{%
\subsubsection{What to turn in}\label{what-to-turn-in}}

\begin{itemize}
\item
  Your Jupyter notebook (or Python script). I must be able to run your
  code - do not turn in a screenshot or code pasted into a Microsoft
  Word document
\item
  The answers to the above questions
\end{itemize}


\end{document}