\documentclass[]{article}
\usepackage{lmodern}
\usepackage{amssymb,amsmath}
\usepackage{ifxetex,ifluatex}
\usepackage{fixltx2e} % provides \textsubscript
\ifnum 0\ifxetex 1\fi\ifluatex 1\fi=0 % if pdftex
  \usepackage[T1]{fontenc}
  \usepackage[utf8]{inputenc}
\else % if luatex or xelatex
  \ifxetex
    \usepackage{mathspec}
  \else
    \usepackage{fontspec}
  \fi
  \defaultfontfeatures{Ligatures=TeX,Scale=MatchLowercase}
\fi
% use upquote if available, for straight quotes in verbatim environments
\IfFileExists{upquote.sty}{\usepackage{upquote}}{}
% use microtype if available
\IfFileExists{microtype.sty}{%
\usepackage{microtype}
\UseMicrotypeSet[protrusion]{basicmath} % disable protrusion for tt fonts
}{}
\usepackage[margin=1in]{geometry}
\usepackage{hyperref}
\PassOptionsToPackage{usenames,dvipsnames}{color} % color is loaded by hyperref
\hypersetup{unicode=true,
            pdftitle={Lab 08},
            colorlinks=true,
            linkcolor=Maroon,
            citecolor=Blue,
            urlcolor=Blue,
            breaklinks=true}
\urlstyle{same}  % don't use monospace font for urls
\IfFileExists{parskip.sty}{%
\usepackage{parskip}
}{% else
\setlength{\parindent}{0pt}
\setlength{\parskip}{6pt plus 2pt minus 1pt}
}
\setlength{\emergencystretch}{3em}  % prevent overfull lines
\providecommand{\tightlist}{%
  \setlength{\itemsep}{0pt}\setlength{\parskip}{0pt}}
\setcounter{secnumdepth}{0}
% Redefines (sub)paragraphs to behave more like sections
\ifx\paragraph\undefined\else
\let\oldparagraph\paragraph
\renewcommand{\paragraph}[1]{\oldparagraph{#1}\mbox{}}
\fi
\ifx\subparagraph\undefined\else
\let\oldsubparagraph\subparagraph
\renewcommand{\subparagraph}[1]{\oldsubparagraph{#1}\mbox{}}
\fi

%%% Use protect on footnotes to avoid problems with footnotes in titles
\let\rmarkdownfootnote\footnote%
\def\footnote{\protect\rmarkdownfootnote}


  \title{Lab 08}
    \author{}
    \date{}
  

% change section title styling
\usepackage{sectsty}
\sectionfont{\normalsize\normalfont\itshape}
\subsectionfont{\normalsize\normalfont}

% use fancyhdr style
\usepackage{fancyhdr}
\pagestyle{fancy}
\fancyhead[LO, LE]{Lab 08}
\fancyhead[RO, RE]{GEOG 432/832}
\makeatletter
\renewcommand{\maketitle}{\bgroup\vspace*{-1cm}\setlength{\parindent}{0pt}
\begin{flushleft}
  \@author
  
  \@date
  
\end{flushleft}\egroup
}
\makeatother

\begin{document}
\maketitle

\hypertarget{lab-08-indicators-of-spatial-autocorrelation}{%
\section{Lab 08: Indicators of spatial
autocorrelation}\label{lab-08-indicators-of-spatial-autocorrelation}}

\hypertarget{read-the-instructions-completely-before-starting-the-lab}{%
\subsubsection{Read the instructions COMPLETELY before starting the
lab}\label{read-the-instructions-completely-before-starting-the-lab}}

In this lab, you will calculate global and local indicators of spatial
autocorrelation. This process includes developing spatial weights
matrices and working with spatially-lagged variables (which you
completed in the previous lab). It will also require you link multiple
``hands on'' portions of in-class work such that you demonstrate your
understanding of underlying concepts and theory.

\hypertarget{tasks}{%
\subsubsection{Tasks:}\label{tasks}}

\begin{enumerate}
\def\labelenumi{\arabic{enumi}.}
\tightlist
\item
  Using the dataset found in unit12inclassdata.zip on Canvas, select a
  variable of interest. As discussed during lecture, real-world data is
  often messy. To create your the shapefile, I started with the
  \textbf{ACS\_10\_5YR\_County} dataset found at
  \url{https://www.census.gov/geographies/mapping-files/2010/geo/tiger-data.html}
  -\textgreater{} 2010 Census -\textgreater{} 2006 - 2010 Detailed
  Tables -\textgreater{} Counties. The direct link can be found here:
  \url{http://www2.census.gov/geo/tiger/TIGER_DP/2010ACS/2010_ACS_5YR_County.gdb.zip}
\end{enumerate}

Because there were a VERY large number of attributes (columns) I further
subsetted the data prior to exporting FROM a feature class in a
GeoDatabase TO a shapefile. During this export, ArcGIS renamed many of
the fields because the shapefile specification only allows attribute
names to be 10 characters long or less. The fields I exported include
the range from ``DP3\_HC01\_VC04'' to ``DP3\_HC04\_VC93'' (original
names). You will need to link the OLD field name (also found in the CSV)
to the new ESRI-generated sequential scheme, as we discussed in class. I
understand this step may be confusing at first, but it is representative
of real-world problem solving with often messy data. Try some of the
strategies we discussed in class. If you have questions, please contact
me EARLY.

\begin{enumerate}
\def\labelenumi{\arabic{enumi}.}
\setcounter{enumi}{1}
\item
  Create a spatial subset of the US, with at AT MINIMUM 4 states,
  MAXIMUM 7 states. States must be contiguous
\item
  Develop a distance-based spatial weights matrix. Do all necessary and
  proper preparatory work (see previous labs and lectures if you need a
  refresher)
\item
  Make a histogram of your chosen variable
\item
  Make a choropleth map of your chosen variable. Choose an appropriate
  data classification scheme
\item
  Plot a kernel density estimation plot with the breaks included
\item
  Make Moran Plot of your chosen variable
\item
  Calculate Moran's I for your dataset
\item
  Calculate local indicators of spatial autocorrelation (LISAs) for your
  chosen variable. Make a map that shows statistically significant HH,
  HL, LH, and LL observations and shades them accordingly
\end{enumerate}

\hypertarget{questions}{%
\subsubsection{Questions:}\label{questions}}

\begin{enumerate}
\def\labelenumi{\arabic{enumi}.}
\item
  Describe in your own words \emph{how Moran's I is calculated}
\item
  Explain and defend your formalization of W. Why is the representation
  of space that you chose appropriate for your analysis? Make a
  convincing case.
\item
  Reflect on the 8 labs from this semester. How does your skillset
  differ from when you started? How does using open source tools compare
  to your experience in ArcGIS Pro and ArcPy?
\end{enumerate}

\hypertarget{what-to-turn-in}{%
\subsubsection{What to turn in}\label{what-to-turn-in}}

\begin{itemize}
\item
  Your Jupyter notebook (do not turn it in as a Python script). I must
  be able to run your code - do not turn in a screenshot or code pasted
  into a Microsoft Word document
\item
  The answers to the above questions
\end{itemize}


\end{document}