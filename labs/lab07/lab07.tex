\documentclass[]{article}
\usepackage{lmodern}
\usepackage{amssymb,amsmath}
\usepackage{ifxetex,ifluatex}
\usepackage{fixltx2e} % provides \textsubscript
\ifnum 0\ifxetex 1\fi\ifluatex 1\fi=0 % if pdftex
  \usepackage[T1]{fontenc}
  \usepackage[utf8]{inputenc}
\else % if luatex or xelatex
  \ifxetex
    \usepackage{mathspec}
  \else
    \usepackage{fontspec}
  \fi
  \defaultfontfeatures{Ligatures=TeX,Scale=MatchLowercase}
\fi
% use upquote if available, for straight quotes in verbatim environments
\IfFileExists{upquote.sty}{\usepackage{upquote}}{}
% use microtype if available
\IfFileExists{microtype.sty}{%
\usepackage{microtype}
\UseMicrotypeSet[protrusion]{basicmath} % disable protrusion for tt fonts
}{}
\usepackage[margin=1in]{geometry}
\usepackage{hyperref}
\PassOptionsToPackage{usenames,dvipsnames}{color} % color is loaded by hyperref
\hypersetup{unicode=true,
            pdftitle={Lab 07},
            colorlinks=true,
            linkcolor=Maroon,
            citecolor=Blue,
            urlcolor=Blue,
            breaklinks=true}
\urlstyle{same}  % don't use monospace font for urls
\IfFileExists{parskip.sty}{%
\usepackage{parskip}
}{% else
\setlength{\parindent}{0pt}
\setlength{\parskip}{6pt plus 2pt minus 1pt}
}
\setlength{\emergencystretch}{3em}  % prevent overfull lines
\providecommand{\tightlist}{%
  \setlength{\itemsep}{0pt}\setlength{\parskip}{0pt}}
\setcounter{secnumdepth}{0}
% Redefines (sub)paragraphs to behave more like sections
\ifx\paragraph\undefined\else
\let\oldparagraph\paragraph
\renewcommand{\paragraph}[1]{\oldparagraph{#1}\mbox{}}
\fi
\ifx\subparagraph\undefined\else
\let\oldsubparagraph\subparagraph
\renewcommand{\subparagraph}[1]{\oldsubparagraph{#1}\mbox{}}
\fi

%%% Use protect on footnotes to avoid problems with footnotes in titles
\let\rmarkdownfootnote\footnote%
\def\footnote{\protect\rmarkdownfootnote}


  \title{Lab 07}
    \author{}
    \date{}
  

% change section title styling
\usepackage{sectsty}
\sectionfont{\normalsize\normalfont\itshape}
\subsectionfont{\normalsize\normalfont}

% use fancyhdr style
\usepackage{fancyhdr}
\pagestyle{fancy}
\fancyhead[LO, LE]{Lab 07}
\fancyhead[RO, RE]{GEOG 432/832}
\makeatletter
\renewcommand{\maketitle}{\bgroup\vspace*{-1cm}\setlength{\parindent}{0pt}
\begin{flushleft}
  \@author
  
  \@date
  
\end{flushleft}\egroup
}
\makeatother

\begin{document}
\maketitle

\hypertarget{lab-07-building-spatial-weights-matrices}{%
\section{Lab 07: Building spatial weights
matrices}\label{lab-07-building-spatial-weights-matrices}}

\hypertarget{read-the-instructions-completely-before-starting-the-lab}{%
\subsubsection{Read the instructions COMPLETELY before starting the
lab}\label{read-the-instructions-completely-before-starting-the-lab}}

This lab will help you become more comfortable with developing spatial
weights matrices and working with spatially-lagged variables. It will
also require you link multiple ``hands on'' portions of in-class work
such that you demonstrate your understanding of underlying concepts and
theory.

\hypertarget{tasks}{%
\subsubsection{Tasks:}\label{tasks}}

\begin{enumerate}
\def\labelenumi{\arabic{enumi}.}
\item
  Select one of the datasets from the libpysal library (see
  \url{https://pysal.org/libpysal/notebooks/examples.html} for details).
  The dataset should include a polygon feature class, as you will make a
  choropleth map in a later task. It should also contain an
  ``interesting'' variable that you will use in a Moran plot.
\item
  If it is not projected, reproject it to an appropriate CRS. Show the
  CRS in the notebook.
\item
  Develop a contiguity-based spatial weights matrix of your choosing
  (i.e., rook or queen)

  \begin{enumerate}
  \def\labelenumii{\arabic{enumii}.}
  \tightlist
  \item
    Plot a histogram of the number of neighbors
  \item
    ``Print'' to the notebook the average number of neighbors
  \item
    Choose one areal unit of interest (e.g., a county), make a plot of
    the unit and its neighbors, similar to what we completed in class.
    NOTE, you MUST use a different (but appropriate) color scheme than
    what we used in class
  \item
    Row-standardized the \emph{W}
  \item
    Using a variable of interest in your dataset, calcuate the a lagged
    variable
  \item
    Make a Moran Plot
  \end{enumerate}
\item
  Repeat \#3 above with a W developed using the distance-band method
\item
  Using the dataset from \#4 (the distance-band-based W), write a script
  that:

  \begin{enumerate}
  \def\labelenumii{\arabic{enumii}.}
  \tightlist
  \item
    Finds those observations in the H-H category
  \item
    Plots a choropleth map of your dataset with a categorical color
    scheme, where the shading corresponds to the Moran plot (really,
    ``LISA'') quadrants. Thus, your map will have four shades of color.
  \end{enumerate}
\end{enumerate}

\emph{Note, you have been provided all the information/knowledge/skills
required to accomplish this task. THINK about how the lagged variables,
the Moran plot, and quadrants work from a conceptual perspective FIRST.
Then write your code}

HINT: \texttt{numpy.where()} is a useful function for generating a new
attribute based on a conditional test

\hypertarget{questions}{%
\subsubsection{Questions:}\label{questions}}

\begin{enumerate}
\def\labelenumi{\arabic{enumi}.}
\item
  Describe in your own words: \emph{what is a spatially-lagged
  variable?}
\item
  How does your analysis in this lab (as simple as it is) differ by how
  you have formalized W (e.g., space, neighbors) in two different
  methods? How might it affect analysis?
\item
  What does it mean if an observation falls in the ``H-L'' quadrant? Why
  might it be useful to detect such occurances?
\item
  Walk me through your process in completing task 5 above. What
  challenges did you encounter and how did you overcome them?
\end{enumerate}

\hypertarget{what-to-turn-in}{%
\subsubsection{What to turn in}\label{what-to-turn-in}}

\begin{itemize}
\item
  Your Jupyter notebook (or Python script). I must be able to run your
  code - do not turn in a screenshot or code pasted into a Microsoft
  Word document
\item
  The answers to the above questions
\end{itemize}


\end{document}