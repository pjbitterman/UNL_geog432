\documentclass[]{article}
\usepackage{lmodern}
\usepackage{amssymb,amsmath}
\usepackage{ifxetex,ifluatex}
\usepackage{fixltx2e} % provides \textsubscript
\ifnum 0\ifxetex 1\fi\ifluatex 1\fi=0 % if pdftex
  \usepackage[T1]{fontenc}
  \usepackage[utf8]{inputenc}
\else % if luatex or xelatex
  \ifxetex
    \usepackage{mathspec}
  \else
    \usepackage{fontspec}
  \fi
  \defaultfontfeatures{Ligatures=TeX,Scale=MatchLowercase}
\fi
% use upquote if available, for straight quotes in verbatim environments
\IfFileExists{upquote.sty}{\usepackage{upquote}}{}
% use microtype if available
\IfFileExists{microtype.sty}{%
\usepackage{microtype}
\UseMicrotypeSet[protrusion]{basicmath} % disable protrusion for tt fonts
}{}
\usepackage[margin=1in]{geometry}
\usepackage{hyperref}
\PassOptionsToPackage{usenames,dvipsnames}{color} % color is loaded by hyperref
\hypersetup{unicode=true,
            pdftitle={Lab 06},
            colorlinks=true,
            linkcolor=Maroon,
            citecolor=Blue,
            urlcolor=Blue,
            breaklinks=true}
\urlstyle{same}  % don't use monospace font for urls
\IfFileExists{parskip.sty}{%
\usepackage{parskip}
}{% else
\setlength{\parindent}{0pt}
\setlength{\parskip}{6pt plus 2pt minus 1pt}
}
\setlength{\emergencystretch}{3em}  % prevent overfull lines
\providecommand{\tightlist}{%
  \setlength{\itemsep}{0pt}\setlength{\parskip}{0pt}}
\setcounter{secnumdepth}{0}
% Redefines (sub)paragraphs to behave more like sections
\ifx\paragraph\undefined\else
\let\oldparagraph\paragraph
\renewcommand{\paragraph}[1]{\oldparagraph{#1}\mbox{}}
\fi
\ifx\subparagraph\undefined\else
\let\oldsubparagraph\subparagraph
\renewcommand{\subparagraph}[1]{\oldsubparagraph{#1}\mbox{}}
\fi

%%% Use protect on footnotes to avoid problems with footnotes in titles
\let\rmarkdownfootnote\footnote%
\def\footnote{\protect\rmarkdownfootnote}


  \title{Lab 06}
    \author{}
    \date{}
  

% change section title styling
\usepackage{sectsty}
\sectionfont{\normalsize\normalfont\itshape}
\subsectionfont{\normalsize\normalfont}

% use fancyhdr style
\usepackage{fancyhdr}
\pagestyle{fancy}
\fancyhead[LO, LE]{Lab 06}
\fancyhead[RO, RE]{GEOG 432/832}
\makeatletter
\renewcommand{\maketitle}{\bgroup\vspace*{-1cm}\setlength{\parindent}{0pt}
\begin{flushleft}
  \@author
  
  \@date
  
\end{flushleft}\egroup
}
\makeatother

\begin{document}
\maketitle

\hypertarget{lab-06-exploratory-spatial-data-analysis-and-visualization}{%
\section{Lab 06: Exploratory spatial data analysis and
visualization}\label{lab-06-exploratory-spatial-data-analysis-and-visualization}}

\hypertarget{read-the-instructions-completely-before-starting-the-lab}{%
\subsubsection{Read the instructions COMPLETELY before starting the
lab}\label{read-the-instructions-completely-before-starting-the-lab}}

This lab is intended to further develop your exploratory spatial data
anlaysis skills, including visualization. This lab builds upon your
in-class activities from this week, which themselves are inspired by the
ENVS 363/563 course at the University of Liverpool (see
\url{https://darribas.org/gds_course/content/bC/lab_C.html}). This lab
ALSO gives you significant latitude to use datasets, methosds, and
visualizations of your own choosing and interests. There are limited
tasks for this lab.

\hypertarget{tasks}{%
\subsubsection{Tasks:}\label{tasks}}

\begin{enumerate}
\def\labelenumi{\arabic{enumi}.}
\item
  Select one of the datasets from the libpysal library (see
  \url{https://pysal.org/libpysal/notebooks/examples.html} for details).
  The dataset should include a polygon feature class, as you will make a
  choropleth map in a later task
\item
  If it is not projected, assign it a proper projection. Show this
  projection in the notebook.
\item
  Perform ESDA on the dataset, including at a minimum:

  \begin{itemize}
  \tightlist
  \item
    Basic summary statistics on at least one relevant attribute
  \item
    Plot of numeric data (e.g., histogram, scatter plot, NOT A MAP)
  \item
    A plot of the geometry, styled in some way that is
    \emph{appropriate} (e.g., alpha, colour)
  \end{itemize}
\item
  Make a choropleth map of a \emph{sensible} variable in your dataset.
  Choose an appropriate data classification scheme
\item
  Plot a kernel density estimation plot with the breaks included (see
  the Jupyter notbook that I provided on Canvas for examples)
\item
  Find ANOTHER dataset from the web \emph{that overlaps spatially with
  your dataset}. For example, a streets file for the city of Chicago
\item
  Make another map, this time with multiple layers. The base layer is
  the choropleth map from above. The top layer is a visualization of the
  datset you found. Give it a title.
\end{enumerate}

\hypertarget{questions}{%
\subsubsection{Questions:}\label{questions}}

\begin{enumerate}
\def\labelenumi{\arabic{enumi}.}
\item
  Describe your dataset and the ESDA you performed. What did you learn?
\item
  What classification scheme did you use? How many classes? Why?
\item
  In your final multi-layer map, why did you choose your particular
  symbolization method?
\end{enumerate}

\hypertarget{what-to-turn-in}{%
\subsubsection{What to turn in}\label{what-to-turn-in}}

\begin{itemize}
\item
  Your Jupyter notebook (or Python script). I must be able to run your
  code - do not turn in a screenshot or code pasted into a Microsoft
  Word document
\item
  The answers to the above questions
\end{itemize}


\end{document}