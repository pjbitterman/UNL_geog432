\documentclass[]{article}
\usepackage{lmodern}
\usepackage{amssymb,amsmath}
\usepackage{ifxetex,ifluatex}
\usepackage{fixltx2e} % provides \textsubscript
\ifnum 0\ifxetex 1\fi\ifluatex 1\fi=0 % if pdftex
  \usepackage[T1]{fontenc}
  \usepackage[utf8]{inputenc}
\else % if luatex or xelatex
  \ifxetex
    \usepackage{mathspec}
  \else
    \usepackage{fontspec}
  \fi
  \defaultfontfeatures{Ligatures=TeX,Scale=MatchLowercase}
\fi
% use upquote if available, for straight quotes in verbatim environments
\IfFileExists{upquote.sty}{\usepackage{upquote}}{}
% use microtype if available
\IfFileExists{microtype.sty}{%
\usepackage{microtype}
\UseMicrotypeSet[protrusion]{basicmath} % disable protrusion for tt fonts
}{}
\usepackage[margin=1in]{geometry}
\usepackage{hyperref}
\PassOptionsToPackage{usenames,dvipsnames}{color} % color is loaded by hyperref
\hypersetup{unicode=true,
            pdftitle={Lab 03},
            colorlinks=true,
            linkcolor=Maroon,
            citecolor=Blue,
            urlcolor=Blue,
            breaklinks=true}
\urlstyle{same}  % don't use monospace font for urls
\IfFileExists{parskip.sty}{%
\usepackage{parskip}
}{% else
\setlength{\parindent}{0pt}
\setlength{\parskip}{6pt plus 2pt minus 1pt}
}
\setlength{\emergencystretch}{3em}  % prevent overfull lines
\providecommand{\tightlist}{%
  \setlength{\itemsep}{0pt}\setlength{\parskip}{0pt}}
\setcounter{secnumdepth}{0}
% Redefines (sub)paragraphs to behave more like sections
\ifx\paragraph\undefined\else
\let\oldparagraph\paragraph
\renewcommand{\paragraph}[1]{\oldparagraph{#1}\mbox{}}
\fi
\ifx\subparagraph\undefined\else
\let\oldsubparagraph\subparagraph
\renewcommand{\subparagraph}[1]{\oldsubparagraph{#1}\mbox{}}
\fi

%%% Use protect on footnotes to avoid problems with footnotes in titles
\let\rmarkdownfootnote\footnote%
\def\footnote{\protect\rmarkdownfootnote}


  \title{Lab 03}
    \author{}
    \date{}
  

% change section title styling
\usepackage{sectsty}
\sectionfont{\normalsize\normalfont\itshape}
\subsectionfont{\normalsize\normalfont}

% use fancyhdr style
\usepackage{fancyhdr}
\pagestyle{fancy}
\fancyhead[LO, LE]{Lab 03}
\fancyhead[RO, RE]{GEOG 432/832}
\makeatletter
\renewcommand{\maketitle}{\bgroup\vspace*{-1cm}\setlength{\parindent}{0pt}
\begin{flushleft}
  \@author
  
  \@date
  
\end{flushleft}\egroup
}
\makeatother

\begin{document}
\maketitle

\hypertarget{lab-03-spatial-and-tabular-analysis-in-arcgis-pro}{%
\section{Lab 03: Spatial and tabular analysis in ArcGIS
Pro}\label{lab-03-spatial-and-tabular-analysis-in-arcgis-pro}}

\hypertarget{read-the-instructions-completely-before-starting-the-lab}{%
\subsubsection{Read the instructions COMPLETELY before starting the
lab}\label{read-the-instructions-completely-before-starting-the-lab}}

This lab builds on many of the discussions and exercises from class.
There are no introductory exercises or background information. Complete
the tasks below and answer the questions that follow. As always, there
are multiple ways accomplishing the task(s) at hand.

The following tasks use the feature classes found in the
``lab03data.zip'' file on Canvas.

\begin{itemize}
\tightlist
\item
  nebraska\_counties\_census.shp: a county-level dataset from the
  USCensus. Contains population and income information
\item
  nhdplus.shp: a hydrologic dataset from the National Hydrograpic
  Dataset at the scale of ``NHDPlus Segments''. The spatial extent is
  the Lake Champlain Basin in Vermont. Relevant fields are as follows:

  \begin{itemize}
  \tightlist
  \item
    urban\_yield: total phosphorus from urban lands in NHDPlus segment
    in units of kg/ha/year
  \item
    ag\_yield: total phosphorus from agricultural lands in NHDPlus
    segment in units of kg/ha/year
  \item
    forest\_yield: total phosphorus from forested lands in NHDPlus
    segment in units of kg/ha/year
  \end{itemize}
\item
  lake\_champlain.shp: Lake Champlain
\end{itemize}

\hypertarget{task-1-analyze-a-hydrologic-dataset}{%
\subsection{Task 1: Analyze a hydrologic
dataset}\label{task-1-analyze-a-hydrologic-dataset}}

1.1. Write a Python script that finds the NHDPlus segment with the
\emph{second} largest \emph{absolute difference} between phosphorus
yield from agricultural land use and phosphorus yield from forested land
use. Your script must use a SearchCursor.

1.2. Find the top 10 NHDPlus segments by total phosphorus yield (yield
is nutrients per unit area per unit time {[}kg/ha/yr{]}). Export the
data to your analysis tool/software of choice (e.g., Excel, R). Create a
scatterplot (e.g., \url{https://en.wikipedia.org/wiki/Scatter_plot})
with agricultural yield on the x-axis and forested yield on the y-axis.
You must use a SearchCursor in your script.

1.3. Make another scatterplot, this time with all NHDPlus segments. Plot
the distance from each NHDPlus segment to Lake Champlain on the x-axis,
and plot total phosphorus yield on the y-axis.

\hypertarget{task-2-nebraska-counties-census-data}{%
\subsection{Task 2: Nebraska counties census
data}\label{task-2-nebraska-counties-census-data}}

2.1 Using an UpdateCursor, delete all counties that do NOT start with
the letter ``L''. Print the names of the remaining counties and their
population of female residents under the age of 5.

2.2 The next task is to calculate an index of social vulnerability. For
the sake of this exercise, we will calculate a very simple vulnerability
index with only 2 components \textbf{(note, vulnerability is MUCH, MUCH
more complex)}. We will use the indicators: \textbf{income}
(``PerCapInc'') and the \textbf{proportion vacant homes} (``Vacant'' /
``TotalUnits''). We will assume vulnerability is \emph{negatively}
correlated with income, and \emph{positively} correlated with vacancy.
This means you will need to account for different directions in the two
indicators. Further, we will assume our initial analysis has shown that
income is \emph{twice as strong of a predictor} of vulnerability than
the vacancy rate.

\textbf{Your job:} develop a metric of ``vulnerability'' using these two
indicators. Map the result using a choropleth map. Take your time to
think through what you're measuring and how. This is where a paper
sketch or pseudocode can be very effective.

\emph{(Note, I am NOT concerned that you build the perfect indicator or
even understand the concept of social vulnerability. I want to see that
you can think through indicators of different magnitude, order, and
importance.)}

\newpage

\hypertarget{what-to-turn-in}{%
\subsection{What to turn in:}\label{what-to-turn-in}}

\begin{itemize}
\item
  Task 1: your code for 1.1, 1.2, and 1.3, scatterplots from 1.2 and 1.3
\item
  Task 2: your code for 2.1, 2.2, your map from 2.2
\end{itemize}

\hypertarget{answer-to-the-following-questions}{%
\subsubsection{Answer to the following
questions:}\label{answer-to-the-following-questions}}

Q1. Are there any observable trends in your scatterplots in 1.2 and 1.3?
If so, what are they?

Q1. For task 2.2, describe how you chose to construct your indicator.
What problems did you encounter and how did you solve them? You must
address: 1) how you handled the different direction of effects on
vulnerability, 2) how you calculated different importance (or
``weights'') of the two metrics, and 3) how you handled the fact the two
metrics are in different units? Again, there is no right or wrong answer
(I know you're not an expert on vulnerability or indicator
construction). Rather, I want you to demonstrate thoughtfulness in your
design process.


\end{document}