\documentclass[]{article}
\usepackage{lmodern}
\usepackage{amssymb,amsmath}
\usepackage{ifxetex,ifluatex}
\usepackage{fixltx2e} % provides \textsubscript
\ifnum 0\ifxetex 1\fi\ifluatex 1\fi=0 % if pdftex
  \usepackage[T1]{fontenc}
  \usepackage[utf8]{inputenc}
\else % if luatex or xelatex
  \ifxetex
    \usepackage{mathspec}
  \else
    \usepackage{fontspec}
  \fi
  \defaultfontfeatures{Ligatures=TeX,Scale=MatchLowercase}
\fi
% use upquote if available, for straight quotes in verbatim environments
\IfFileExists{upquote.sty}{\usepackage{upquote}}{}
% use microtype if available
\IfFileExists{microtype.sty}{%
\usepackage{microtype}
\UseMicrotypeSet[protrusion]{basicmath} % disable protrusion for tt fonts
}{}
\usepackage[margin=1in]{geometry}
\usepackage{hyperref}
\PassOptionsToPackage{usenames,dvipsnames}{color} % color is loaded by hyperref
\hypersetup{unicode=true,
            pdftitle={Lab 04},
            colorlinks=true,
            linkcolor=Maroon,
            citecolor=Blue,
            urlcolor=Blue,
            breaklinks=true}
\urlstyle{same}  % don't use monospace font for urls
\IfFileExists{parskip.sty}{%
\usepackage{parskip}
}{% else
\setlength{\parindent}{0pt}
\setlength{\parskip}{6pt plus 2pt minus 1pt}
}
\setlength{\emergencystretch}{3em}  % prevent overfull lines
\providecommand{\tightlist}{%
  \setlength{\itemsep}{0pt}\setlength{\parskip}{0pt}}
\setcounter{secnumdepth}{0}
% Redefines (sub)paragraphs to behave more like sections
\ifx\paragraph\undefined\else
\let\oldparagraph\paragraph
\renewcommand{\paragraph}[1]{\oldparagraph{#1}\mbox{}}
\fi
\ifx\subparagraph\undefined\else
\let\oldsubparagraph\subparagraph
\renewcommand{\subparagraph}[1]{\oldsubparagraph{#1}\mbox{}}
\fi

%%% Use protect on footnotes to avoid problems with footnotes in titles
\let\rmarkdownfootnote\footnote%
\def\footnote{\protect\rmarkdownfootnote}


  \title{Lab 04}
    \author{}
    \date{}
  

% change section title styling
\usepackage{sectsty}
\sectionfont{\normalsize\normalfont\itshape}
\subsectionfont{\normalsize\normalfont}

% use fancyhdr style
\usepackage{fancyhdr}
\pagestyle{fancy}
\fancyhead[LO, LE]{Lab 04}
\fancyhead[RO, RE]{GEOG 432/832}
\makeatletter
\renewcommand{\maketitle}{\bgroup\vspace*{-1cm}\setlength{\parindent}{0pt}
\begin{flushleft}
  \@author
  
  \@date
  
\end{flushleft}\egroup
}
\makeatother

\begin{document}
\maketitle

\hypertarget{lab-04-geometries-spatial-queries-and-raster-data-set}{%
\section{Lab 04: Geometries, spatial queries, and raster data
set}\label{lab-04-geometries-spatial-queries-and-raster-data-set}}

\hypertarget{read-the-instructions-completely-before-starting-the-lab}{%
\subsubsection{Read the instructions COMPLETELY before starting the
lab}\label{read-the-instructions-completely-before-starting-the-lab}}

This lab is the final exercise using ArcGIS Pro and ArcPy. It builds on
many of the discussions and exercises from class and will require you to
draw connections among the units we have covered this far. There are no
introductory exercises or background information. Complete the tasks
below and answer the questions that follow. As always, there are
multiple ways accomplishing the task(s) at hand.

The following tasks use the feature classes and raster datasets found in
the ``lab04data.zip'' file on Canvas.

\begin{itemize}
\tightlist
\item
  State\_Park\_Locations.shp: state park locations (points) in Nebraska
\item
  Streams\_303\_d\_.shp: Nebraska streams that are impaired according to
  Section 303d of the Clean Water Act
\item
  Municipal\_Boundaries: The boundaries for cities and towns in Nebraska
\item
  lancaster\_county.shp: Lancaster County, NE
\item
  nlcd\_lc\_14n.tif: land cover for Lancaster County. From the National
  Land Cover Dataset
\item
  ned30lc.tif: a digitial elevation model for Lancaster County. From the
  National Elevation Dataset
\end{itemize}

\hypertarget{task-1-geometries-and-spatial-queries}{%
\subsection{Task 1: Geometries and spatial
queries}\label{task-1-geometries-and-spatial-queries}}

Write a Python script that does all of the following:

\begin{itemize}
\tightlist
\item
  For all state parks within 3 miles of a municipal boundary
\item
  Prints:

  \begin{itemize}
  \tightlist
  \item
    The name of the park
  \item
    Size in acres formatted to 2 decimal places
  \item
    The name of the nearest municipality
  \item
    BONUS POINTS: include whether the park is \emph{more than} 10 miles
    from a 303d stream
  \end{itemize}
\end{itemize}

Example format:

\begin{verbatim}
Arbor Lodge SHP is 56.79 acres, is nearest to Nebraska City, 
and is not more than 10 miles from a 303d stream
\end{verbatim}

NOTE: At minimum you must use a spatial query and a search cursor.

\hypertarget{task-2-mock-suitability-analysis}{%
\subsection{Task 2: Mock suitability
analysis}\label{task-2-mock-suitability-analysis}}

The blue-footed great plains jackalope is rarely sighted anymore and has
thus been placed on the the state's protected species list. It is your
job to find the suitable habitat for the species within Lancaster
County. Suitable habitat includes ALL of the following:

\begin{itemize}
\tightlist
\item
  Within 2 km of a state park
\item
  1 km or more away from a 303d stream
\item
  More than 5 km from municipal boundaries
\item
  More than 401 meters in elevation
\item
  On areas of ``moderate'' slope. I leave you to develop your own
  classification scheme for slope
\end{itemize}

Make a map of the suitable habitat for the blue-footed great plains
jackalope using proper cartographic practices

\newpage

\hypertarget{what-to-turn-in}{%
\subsection{What to turn in:}\label{what-to-turn-in}}

\begin{itemize}
\item
  Task 1: your code
\item
  Task 2: your code, your map
\end{itemize}

\hypertarget{answers-to-the-following-questions}{%
\subsubsection{Answers to the following
questions:}\label{answers-to-the-following-questions}}

Q1. What challenges did you encounter during this lab? How did you
overcome them?

Q2. What new techniques did you learn during this lab? During the first
``half'' of the semester/labs?

Q3. Provide \emph{constructive} feedback about the labs thus far. For
example: what did you like? Dislike? Were the labs too difficult? Too
easy? Too long?


\end{document}